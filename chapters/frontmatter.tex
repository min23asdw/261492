\maketitle
\makesignature

\ifproject
  \begin{abstractTH}
    \enskip ในปัจจุบันการซื้อสินค้าปลีก ณ จุดจำหน่ายสินค้าทั่วไปในประเทศไทย เช่น ซุปเปอร์มาเก็ต ห้างสรรพสินค้า หรือ ร้านค้าปลีกต่าง ๆ นั้น
    มีการใช้ระบบการจ่ายเงินที่จุดชำระสินค้า หรือแคชเชียร์กับพนักงาน ซึ่งระบบดังกล่าวนั้นไม่ตอบโจทย์กับความต้องการ และความสะดวกสบายของผู้ซื้อสินค้า
    เนื่องจากมีข้อเสีย คือการรอ และต่อแถวเพื่อชำระสินค้า  นอกจากนี้จะเห็นว่าปัจจุบันบริษัทต่าง ๆได้เปลี่ยนรูปแบบการบริการให้ผู้บริโภคสามารถทำธุรกรรมได้
    ผ่านโทรศัพท์มือถือ จึงมีความสมควรที่จะพัฒนาระบบซอร์ฟแวร์ที่เป็นตัวช่วยให้ลูกค้าสมารถซื้อสินค้าได้ด้วยตนเอง โดยไม่ต้องผ่านจุดบริการของร้านค้า

    \enskip    แคปแสนป (CapSnap) เป็นระบบจัดการการซื้อ-ขายสินค้าในร้านค้าปลีก เพื่ออำนวยความสะดวก และลดความยุ่งยากในการ
    ซื้อ-ขายสินค้าทั้งทางฝั่งลูกค้า และฝั่งร้านค้า โดยลูกค้าสามารถจัดการการซื้อสินค้าได้ด้วยตนเองผ่านทางแอปพลิเคชัน
    บนโทรศัพท์มือถือ โดยการสแกนสินค้าที่
    ต้องการแบบเรียลไทม์ ซึ่งระบบจะใช้หลักการทางความฉลาดเชิงคำนวณ (Computational Intelligence) เพื่อบอกรายละเอียดและราคาของสินค้าแต่ละชนิดที่ถูกแสกน
    \enskip ซึ่งจะเป็นประโยชน์ต่อคนที่มีปัญหาในการมองเห็นในการเลือกซื้อสินค้า ลูกค้าสามารถระบุรายการสินค้าได้ด้วยตนเอง
    ผ่านการเพิ่ม หรือลดสินค้าในตะกร้าของแอปพลิเคชัน จากนั้นลูกค้าสามารถทําการชําระเงินได้ด้วยตนเองผ่านแอปพลิเคชัน นอกจากนี้ระบบยังมี
    \enskip เว็บไซต์สำหรับร้านค้า (Website Dashboard) ให้บริการสําหรับฝั่งร้านค้า เพื่อช่วยให้ร้านค้าปลีกสามารถจัดการร้านได้อย่างมีประสิทธิภาพมากขึ้นโดยสามารถจัดการ
    และดูข้อมูลคลังสินค้า รวมถึงข้อมูลการขายสินค้าได้ในเว็บไซต์เดียว

  \end{abstractTH}

  \begin{abstract}
    Nowadays, purchasing retail products at general retail stores in Thailand, such as supermarkets,
    department stores, or small retailers offer payment at the checkout point or cashier,
    which does not satisfy the requirements and convenience of customers because there are disadvantages,
    namely waiting and queuing up to pay. In addition, it can be seen at present that different companies
    have modified their services permitting consumers to conduct transactions using mobile phones. Thus,
    it is appropriate to develop a software system that enables clients to independently purchase things
    and checkout without having to go through the store's service counter.
    In order to simplify and lessen the complexity of purchasing and selling products on both the consumer
    side and the store side, we have designed the CapSnap system for handling purchasing in retail establishments
    via mobile application and allows customers to make their own purchases. Moreover, customers can live-stream product images using the function in the mobile application in real-time.
    The details and costs of each scanned item are then retrieved by the system using a computational intelligence technique,
    which is advantageous for users with visual impairments. The shopping cart feature of the application allows users to
    add and remove products. Subsequently, customers can handle their own payments through the application.
    Eventually, the system offers a Website Dashboard service for the retailer side which enable them to monitor and view selling and stock information,
    which helps them run their stores more effectively.

  \end{abstract}

  \iffalse
    \begin{dedication}
      This document is dedicated to all Chiang Mai University students.

      Dedication page is optional.
    \end{dedication}
  \fi % \iffalse

  % \begin{acknowledgments}
  % Your acknowledgments go here. Make sure it sits inside the
  % \texttt{acknowledgment} environment.

  % \acksign{2020}{5}{25}
  % \end{acknowledgments}%
\fi % \ifproject

\contentspage

\ifproject
  \figurelistpage

  \tablelistpage
\fi % \ifproject

% \abbrlist % this page is optional

% \symlist % this page is optional

% \preface % this section is optional
