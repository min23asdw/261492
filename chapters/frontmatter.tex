\maketitle
\makesignature

\ifproject
\begin{abstractTH}
    \enskip แคปแสนป(CapSnap) เป็นระบบจัดการการซื้อ-ขายสินค้าในร้านค้าปลีก เพื่ออำนวยความสะดวก และลดความยุ่งยากในการ
    ซื้อ-ขายสินค้าทั้งทางฝั่งลูกค้า และฝั่งร้านค้า โดยลูกค้าสามารถจัดการการซื้อสินค้าได้ด้วยตนเองผ่านทางแอปพลิเคชัน
    ในโทรศัพท์มือถือ โดยแอพลิเคชันดังกล่าวจะทำการเชื่อมต่อกับร้านค้าเมื่อลูกค้าเดินเข้าร้านค้า จากนั้นลูกค้าสามารถสแกนสินค้าที่
    ต้องการแบบเรียลไทม์ ซึ่งระบบจะใช้หลักการทาง computational intelligence เพื่อบอกรายละเอียดและราคาของสินค้าแต่ละชนิดที่ถูกแสกน
    \enskip ซึ่งจะเป็นประโยชน์ต่อคนที่มีปัญหาในการมองเห็นในการเลือกซื้อสินค้า ลูกค้าสามารถระบุรายการสินค้าได้ด้วยตนเอง
    ผ่านการเพิ่ม หรือลดสินค้าในตะกร้าของแอพลิเคชัน เมื่อลูกค้าเดินออกจากร้าน ระบบจะทำการชำระเงินให้โดยอัตโนมัติ 
    ซึ่งระบบรองรับการใช้งานแอพลิเคชันกับร้านค้าปลีกหลาย ๆร้านที่มีสินค้าแตกต่างกัน นอกจากนี้ระบบยังมี 
    \enskip Website dashboard ให้บริการสําหรับฝั่งร้านค้า เพื่อช่วยให้ร้านค้าปลีกสามารถจัดการร้านได้อย่างมีประสิทธิภาพมากขึ้นโดยสามารถจัดการ 
    และดูข้อมูลคลังสินค้า รวมถึงข้อมูลการขายสินค้าได้ในเว็บไซต์เดียว
    
\end{abstractTH}

% \begin{abstract}
% The abstract would be placed here. It usually does not exceed 350 words
% long (not counting the heading), and must not take up more than one (1) page
% (even if fewer than 350 words long).

% Make sure your abstract sits inside the \texttt{abstract} environment.
% \end{abstract}

\iffalse
\begin{dedication}
This document is dedicated to all Chiang Mai University students.

Dedication page is optional.
\end{dedication}
\fi % \iffalse

\begin{acknowledgments}
Your acknowledgments go here. Make sure it sits inside the
\texttt{acknowledgment} environment.

\acksign{2020}{5}{25}
\end{acknowledgments}%
\fi % \ifproject

\contentspage

\ifproject
\figurelistpage

\tablelistpage
\fi % \ifproject

% \abbrlist % this page is optional

% \symlist % this page is optional

% \preface % this section is optional
