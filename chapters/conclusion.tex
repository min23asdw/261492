\chapter{\ifenglish Conclusions and Discussions\else บทสรุปและข้อเสนอแนะ\fi}

\section{\ifenglish Conclusions\else สรุปผล\fi}
โดยพบว่าการใช้ Computational Intelligence มาเพื่อช่วยในการ Classification และการ Checkout ด้วยตัวเองของ  ลูกค้าทั่วไปสามารถติดตั้งแอปพลิเคชันบนโทรศัพท์มือถือ
ทำให้เวลาที่ใช้ในการซื้อของ ลดลง และน้อยกว่าการไปต่อแถวที่ Cashier เนื่องจากไม่ต้องเข้าแถวรอคิด สามารถที่จะจ่ายเงินได้ด้วยตัวเองได้เลย

แต่ก็ยังมีข้อผิดผลาดในเรื่องของความแม่นยำ และความเร็วในการ Classification ที่น้อยกว่าการใช้ barcode ที่แปะอยู่ตามสินค้า แต่ละชนิด
โดยจากในส่วนของการ Blind Test  จะเห็นว่าหาก Model มีมุมที่ไม่เคยเห็นก็อาจจะทำให้ไม่สามารถ ที่จะทำนายชนิดของสินค้าได้ถูกต้อง 
ดังตาราง \ref{tab:model_test_performance}

\section{Web Dashboard}
จากผลลัพธ์การทำงานของเว็บไซต์สำหรับร้านค้า สามารถสรุปผลการใช้งานเมื่อเทียบกับการไม่มีระบบเว็บไซต์ดังกล่าวในการช่วยจัดการร้านค้าได้ คือ สามารถติดตามการขายสินค้าได้ตลอดเวลา ไม่ว่าจะเป็นผู้ซื้อ สินค้าที่ถูกซื้อ จำนวนเงินที่ได้ รวมถึงสามารถติดตามรายานยอดขายได้อีกด้วย นอกจากนี้ยังสามารถแก้ไขข้อมูลสินค้าทั้งหมดบนเว็บไซต์ ทำให้ร้านค้ามีความสะดวกสบายมากขึ้นในการขาย และจัดการสินค้า โดยสามารถสรุปผลลัพธ์การใช้งานได้ดังนี้

\begin{table}[htbp]
    \centering
    \caption{เปรียบเทียบการจัดการร้านค้าเมื่อไม่มีและมีระบบเว็บไซต์สำหรับร้านค้า}
    \begin{tabularx}{\textwidth}{ |c|X|X| }
        \hline
        \textbf{}            & \textbf{Without Website Dashboard}         & \textbf{With Website Dashboard} \\
        \hline
        การดูข้อมูลสินค้า         & ดูบนไฟล์ Excel                               & มีการแสดงผลบนเว็บไซต์              \\
        \hline
        การเปลี่ยนแก้ไขข้อมูลสินค้า & แก้ไขบนไฟล์ Excel                            & แก้ไขบนเว็บไซต์                    \\
        \hline
        การติดตามสินค้าที่ขาย     & นับสินค้าที่ขายไปจากจำนวนที่เหลือ และติดตามยอดเงินเอง & มีหน้าสรุปว่าลูกท่านใดซื้ออะไรไปบ้าง     \\
        \hline
        การติดตามยอดขาย       & เขียนสรุป และบันทึกยอดขายเอง                   & มีหน้าสรุปยอดขายในทุกๆวัน            \\
        \hline
        การเพิ่มลดคลังสินค้า      & ไม่สามารถติดตามผู้เพิ่มลดได้                      & มีหน้าสรุปผู้เพิ่มลดตลังสินค้า            \\
        \hline
        การติดตามลูกค้า         & ไม่สามารถติดตามได้                            & ติดตามได้บนเว็บไชต์                 \\
        \hline
    \end{tabularx}
\end{table}
\section{\ifenglish Challenges\else ปัญหาที่พบและแนวทางการแก้ไข\fi}
ในการทำโครงงานนี้ พบว่าเกิดปัญหาหลักๆ ดังนี้

\subsection{Classification model}
\par การ Classification ของ product สามารถเกิดความผิดพลาดได้ เมื่อมีการเคลื่อนที่ของกล้อง หรือการขยับของกล้องเร็วเกินไป

แนวทางการแก้ไขปัญหา:
การใช้โมเดลที่มีความยืดหยุ่นมากขึ้น (Robust Models): เลือกใช้โมเดลที่มีความสามารถในการรับมือกับภาพที่เคลื่อนที่ หรือโมเดลที่ได้รับการฝึกสอนด้วยข้อมูลที่มีการเคลื่อนที่ภายใน
\begin{itemize}
    \item การปรับแต่งพารามิเตอร์ (Parameter Tuning): ปรับค่าพารามิเตอร์ของโมเดล เช่น การปรับค่าความสำคัญของชั้นคอนโวลูชัน (convolutional layer) หรือการเปลี่ยนแปลงการตั้งค่าการฝึกสอนเพื่อให้โมเดลมีประสิทธิภาพต่อการเคลื่อนที่ของกล้อง
    \item การแสดงผลลัพย์ ที่มีความมันใจมากเท่านั้น  (confidence): โดยหาก model ทำนายชนิดสินค้าโดยมีค่าความที่ต่ำ (confidence ไม่สูง) อาจจะทำให้ได้ผลลัพย์ที่ผิดพลาดได้ง่าย
    \item การใช้ผลการ predict หลายๆคร้งมาเพิ่มความแม่นยำ (Voting): โดยใช้การส่งรูปภาพไปหลายๆภาพ และนำผลลัพย์การ predict ในแต่ละครั้งมาหาชนิดสินค้าที่ได้รับการ predict ออกมาเยอะที่สุด
    \item การใช้เทคโนโลยีการตรวจจับการเคลื่อนที่ (Motion Detection Technology): การใช้เทคโนโลยีเชิงภาพเพื่อตรวจจับการเคลื่อนที่ของกล้องและปรับปรุงการทำนายคลาสตามไปด้วย เช่น การใช้เซ็นเซอร์การเคลื่อนที่หรือการตรวจจับการเคลื่อนที่จากภาพ
\end{itemize}

\subsection{Machine learning scalability}
\par ในโครงการนี้ Classification server โดย Model ของเรามีขนาดอยู่ที่ 33 ล้านพารามิเตอร์ (128 MB)
ทำงานบนคอมพิวเตอร์ที่มี GPU GTX 1660Ti (6GB VRAM)  computational power อยู่ที่ FP32 (float) 5.437 TFLOPS.
ซึ่งการทำนาย ของรูปภาพ 1 รูป model จะใช้เวลาอย่างนานที่สุด 25 ms
5.437 TFLOPS ( computational power 5.437 TFLOPS หากใช้ GPU ที่มีพลังคำนวณมากกว่าจะทำให้ใช้เวลาน้อยลง)

โดยเฉลี่ยแล้ว User 1 คน ใช้เวลาในการใช้งาน Camera classification อยู่ที่ 3-10 วินาที
โดย Application จะส่งรูปมายังเซิร์ฟเวอร์ทุก 500 มิลลิวินาที (1วินาที ได้รับ 2 รูปภาพ)

ซึ่งสามารถคิดได้ว่า ใน 1 วินาที จะสงวน slot ที่เท่ากับ 50 มิลลิวินาที เป็นของผู้ใช้ 1 คน
ระบบนี้ก็สามารถรองรับได้ถึง 20 ผู้ใช้ต่อ 1 วินาที
% (1000 ms / 25 ms ) / 2 Image/User = 20 User
\begin{equation}
    ({\frac {1,000 ms}{25 ms / image}} )  / 2 \text{ image / User }  = 20 \text{User }
\end{equation}

ดังนั้น ใน 1 วินาทีซึ่งแบ่งได้ทั้งหมด 20  slot สามารถรองรับผู้ใช้ได้พร้อมกัน 20 คน
และสมมุติว่า User แต่ละคนใช้เวลามากถึง 10 วินาที (20 รูปภาพ) ระบบนี้ก็สามารถรองรับได้ถึง 120 ผู้ใช้ต่อ 1 นาที

% ( 60,000 ms / 25ms ) / 20 Image/User  = 120 user
\begin{equation}
    ({\frac {60,000 ms}{25 ms / image}} )   / 20 \text{ image / User }  = 120 \text{User }
\end{equation}

แนวทางการแก้ไขปัญหา:

\begin{itemize}
    \item เพิ่ม computational power ของ Classification server ให้มากขึ้น
    \item ลดจำนวนรูปภาพที่ Application จะส่งมายัง Classification server ลง
    \item ลด Input Dimension ของรูปภาพ
\end{itemize}

\subsection{Web Dashboard}
ปัญหาที่พบในการในการใช้งานส่วนของเว็บไซต์สำหรับร้านค้า คือเวลาที่ใช้ในการแสดงผลข้อมูลรายงานยอดขายในกรณีที่มีข้อมูลจำนวนมากนั้นใช้เวลาปะรมาณ 5 วินาที
ขึ้นไปในการแสดงผล ซึ่งถือว่าค่อนข้างช้า และให้ประสบการณ์ที่ได้ราบรื่นในการใช้บริการเว็บไซต์ แนวทางในการแก้ไขคือการปรับปรุงอัลกอริทึมในการประมวลผลยอดขายโดยพิจารณาทฤษฎีของ
Big O น้อยที่สุดที่เป็นไปได้ เพื่อให้การประมวลรวดเร็วยิ่งขึ้น


% ในการทำโครงงานนี้ พบว่าเกิดปัญหาหลักๆ ดังนี้

\section{\ifenglish%
      Suggestions and further improvements
  \else%
      ข้อเสนอแนะและแนวทางการพัฒนาต่อ
  \fi
 }
 สำหรับข้อแนะนำสำหรับการพัฒนาต่อของระบบโดยรวม สามารถจำแนกได้ดังนี้

 \begin{itemize}
    \item \textbf{ปรับปรุงระบบคลาวด์:} พัฒนาระบบคลาวด์โดยการอัปเกรด Supabase เพื่อเพิ่มประสิทธิภาพในการทำงานและประหยัดค่าใช้จ่าย เช่น การเพิ่ม Scalability Reliability และ Security ให้กับระบบ
    \item \textbf{ปรับปรุงโมดูลความสามารถของ Classification Model:} พัฒนาโมดูล CI เพื่อให้สามารถจำแนกสินค้าจากรูปภาพได้ที่แม่นยำมากยิ่งขึ้น โดยอาจเพิ่มข้อมูล หรือปรับปรุงโมเดลใหม่
    \item \textbf{เพิ่มฟีเจอร์ใน Mobile Application:} เพิ่มฟีเจอร์ใหม่บนแอปพลิเคชัน เพื่อให้มีการทำงานที่สมบูรณืยิ่งขึ้น เช่น เพิ่มตัวเลือกการลงทะเบียนเข้าใช้งาน เพิ่มการตั้งค่าข้อมูลผู้ใช้งานมากขึ้น ปรับปรุงการแสดงผลประวัติการซื้อสินค้า ปรังปรุงเรื่องความเสถียรของการแสกนสินค้า และเพิ่มการชำระเงินหลากหลายรูปแบบ
    \item \textbf{พัฒนาฟีเจอร์ใน Web Dashboard:} เพิ่มฟีเจอร์การวิเคราะห์ข้อมูลเพิ่มเติม เช่น การวิเคราะห์ยอดขายที่ซับซ้อนมากขึ้น การติดตามแนวโน้มการซื้อของลูกค้า รวมถึงปรับปรุงโปรแกรมให้เร็ว และเสียรมากขึ้น
    \item \textbf{ปรับปรุงฐานข้อมูล:} ปรับปรุงโครงสร้างของฐานข้อมูลเพื่อเพิ่มประสิทธิภาพในการจัดการข้อมูล และการสืบค้น
    \item \textbf{ทดสอบและปรับปรุงความปลอดภัย:} ทำการทดสอบความปลอดภัยของระบบอย่างสม่ำเสมอ และปรับปรุงเพื่อป้องกันการแอบแฝงข้อมูลหรือการโจมตีจากภัยคุกคามต่าง ๆ
    \item \textbf{การพัฒนา UX/UI:} ปรับปรุงประสิทธิภาพ และประสบการณ์ผู้ใช้ในทั้ง Mobile Application และ Web Dashboard เพื่อให้การใช้งานเป็นไปอย่างราบรื่นและสะดวกสบายยิ่งขึ้น 
    \item \textbf{ใช้ Machine Learning ในการคาดเดาความต้องการของลูกค้า:} ใช้ Machine Learning เพื่อวิเคราะห์และคาดเดาความต้องการของลูกค้าเพื่อเสนอสินค้าที่เข้ากับความต้องการของลูกค้าได้ดียิ่งขึ้น
    \item \textbf{เพิ่มระบบการสนับสนุนลูกค้า:} พัฒนาระบบการสนับสนุนลูกค้าเพิ่มเติม เช่น ระบบแชทบอท หรือระบบช่วยเหลืออื่น ๆ เพื่อสนับสนุนลูกค้าในการใช้บริการและแก้ไขปัญหาที่เกิดขึ้น
    \item \textbf{การใช้ Big Data Analytics:} ใช้เทคโนโลยี Big Data Analytics เพื่อวิเคราะห์และสร้างความเข้าใจในพฤติกรรมของลูกค้า เพื่อช่วยในการตัดสินใจในการพัฒนาและปรับปรุงระบบให้เหมาะสมยิ่งขึ้น
\end{itemize}

% ข้อเสนอแนะเพื���อพัฒนาโครงงานนี้ต่อไป มีดังนี้
