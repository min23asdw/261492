\chapter{\ifenglish Conclusions and Discussions\else บทสรุปและข้อเสนอแนะ\fi}

\section{\ifenglish Conclusions\else สรุปผล\fi}

\subsection{Mobile Application }
 
\begin{table}[htbp]
    \centering
    \caption{Time Usage for Checkout Process}
    \begin{tabular}{|p{5cm}|p{2cm}|}
        \hline
        \textbf{Camera Classification} & \textbf{Time Usage (seconds)} \\
        \hline
        Image capture                  & 2 s                           \\
        Image processing               & 3 s                           \\
        Classification result          & 1 s                           \\
        \hline
        Total Time Usage               & 1                             \\
        \hline
    \end{tabular}
\end{table}

\begin{table}[htbp]
    \centering
    \caption{Time Usage for Checkout Process}
    \begin{tabular}{|p{5cm}|p{2cm}|}
        \hline
        \textbf{Payment}     & \textbf{Time Usage (seconds)} \\
        \hline
        Human calculation    & 10 s                          \\
        Payment confirmation & 3 s                           \\
        \hline
        Total Time Usage     & 1                             \\
        \hline
    \end{tabular}
\end{table}


\begin{table}[htbp]
    \centering
    \caption{Time Usage for Checkout Process}
    \begin{tabular}{|p{5cm}|p{2cm}|}
        \textbf{Line Checkout Cashier} & \textbf{Time Usage (seconds)} \\
        \hline
        Waiting in line                & 30-120 s                      \\
        Cashier processing             & 20-40 s                       \\
        Payment transaction            & 10-15 s                       \\
        \hline
        Total Time Usage               & 1                             \\
        \hline
    \end{tabular}
\end{table}
 
\section{\ifenglish Challenges\else ปัญหาที่พบและแนวทางการแก้ไข\fi}
ในการทำโครงงานนี้ พบว่าเกิดปัญหาหลักๆ ดังนี้
\subsection{Classification model}
 

การ Classification ของ product สามารถเกิดความผิดพลาดได้ เมื่อมีการเคลื่อนที่ของกล้อง หรือขยับเกล้องเร็วไป

แนวทางการแก้ไขปัญหานี้สามารถทำได้ :
การใช้โมเดลที่มีความยืดหยุ่นมากขึ้น (Robust Models): เลือกใช้โมเดลที่มีความสามารถในการรับมือกับภาพที่เคลื่อนที่ หรือโมเดลที่ได้รับการฝึกสอนด้วยข้อมูลที่มีการเคลื่อนที่ภายใน

การปรับแต่งพารามิเตอร์ (Parameter Tuning): ปรับค่าพารามิเตอร์ของโมเดล เช่น การปรับค่าความสำคัญของชั้นคอนโวลูชัน (convolutional layer) หรือการเปลี่ยนแปลงการตั้งค่าการฝึกสอนเพื่อให้โมเดลมีประสิทธิภาพต่อการเคลื่อนที่ของกล้อง
การแสดงผลลัพย์ หากมีค่าความมันใจมากเท่านั้น  (confidence): โดยหาก model ทำนายชนิดสินค้าโดยมีค่าความที่ต่ำ (confidence ไม่สูง) อาจจะทำให้ได้ผลลัพย์ที่ผิดพลาดได้ง่าย
การใช้ผลการ predict หลายๆคร้งมาเพื่มความแม่นยำ (Voting): โดยใช้การส่งรูปภาพไปหลายๆภาพ และนำผลลัพย์การ predict ในแต่ละครั้งมาหาชนิดสินค้าที่ได้รับการ predict ออกมาเยอะที่สุด
การใช้เทคโนโลยีการตรวจจับการเคลื่อนที่ (Motion Detection Technology): การใช้เทคโนโลยีเชิงภาพเพื่อตรวจจับการเคลื่อนที่ของกล้องและปรับปรุงการทำนายคลาสตามไปด้วย เช่น การใช้เซ็นเซอร์การเคลื่อนที่หรือการตรวจจับการเคลื่อนที่จากภาพ
 
\subsection{Machine learning scalability}


ในโครงการนี้ Classification server ของเราทำงานบนคอมพิวเตอร์ที่มี GPU GTX 1660Ti (6GB Vram)

ซึ่งมี  computational power อยู่ที่ FP32 (float) 5.437 TFLOPS.
ในการทำนายภาพ 1 รูป เซิร์ฟเวอร์ใช้เวลาอย่างนานที่สุด 25 มิลลิวินาที
5.437 TFLOPS ( computational power 5.437 TFLOPS หากใช้ GPU ที่มีพลังคำนวณมากกว่าจะทำให้ใช้เวลาน้อยลง)

โดยเฉลี่ยแล้ว User 1 คน ใช้เวลาในการใช้งาน Camera classification อยู่ที่ 3-10 วินาที
โดย Application จะส่งรูปมายังเซิร์ฟเวอร์ทุก 500 มิลลิวินาที (1วินาที ได้รับ 2 รูปภาพ)

ซึ่งสามารถคิดได้ว่า ใน 1 วินาที จะสงวน slot ที่เท่ากับ 50 มิลลิวินาที เป็นของผู้ใช้ 1 คน 
ระบบนี้ก็สามารถรองรับได้ถึง 20 ผู้ใช้ต่อ 1 วินาที
% (1000 ms / 25 ms ) / 2 Image/User = 20 User
\begin{equation}
    {\frac {1,000 ms}{25 ms / image}}   / 2 \text{ image / User }  = 20 \text{User }
\end{equation}

ดังนั้น ใน 1 วินาทีซึ่งแบ่งได้ทั้งหมด 20  slot สามารถรองรับผู้ใช้ได้พร้อมกัน 20 คน

และสมมุติว่า User แต่ละคนใช้เวลามากถึง 10 วินาที (20 รูปภาพ) ระบบนี้ก็สามารถรองรับได้ถึง 120 ผู้ใช้ต่อ 1 นาที 

% ( 60,000 ms / 25ms ) / 20 Image/User  = 120 user

\begin{equation}
    {\frac {60,000 ms}{25 ms / image}}   / 20 \text{ image / User }  = 120 \text{User }
\end{equation}
 

\subsection{Web Dashboard}
//TODO 



% ในการทำโครงงานนี้ พบว่าเกิดปัญหาหลักๆ ดังนี้

\section{\ifenglish%
      Suggestions and further improvements
  \else%
      ข้อเสนอแนะและแนวทางการพัฒนาต่อ
  \fi
 }

% ข้อเสนอแนะเพื่อพัฒนาโครงงานนี้ต่อไป มีดังนี้
