\chapter{\ifproject%
      \ifenglish Experimentation and Results\else การทดลองและผลลัพธ์\fi
  \else%
      \ifenglish System Evaluation\else การประเมินระบบ\fi
  \fi}



\section{Classification model}
\par ผลลัพธ์ จากการ train และ validation  Classification model ด้วย 3546 sample เป็นจำนวน 200 Epoch
ทั้งหมด 4 Fold


\par ตัวอย่าง ผลการทดลองของการ validation ด้วย 3546 sample ในแต่ละ Fold
\begin{figure}[h]
    \begin{center}

        \includegraphics[scale=0.4]{pic/model/train_fold.png}
    \end{center}

    \caption[Train fold]{Train fold}
    \label{fig:Train fold}
\end{figure}

% โดยเมื่อนำ evaluate มาหา cross_entropy 0.23343226313591003, 0.944695234298706 accuracy
จากทั้ง 4 Fold ทำการเลือก model ที่มีค่า Validation Score สูงสุดจากใน 4 model
โดยนำ model นี้ไปทำการ Evaluate ด้วย Dataset ที่ไม่เคยเห็น (Blind Test)


\begin{align}
    \text{cross\_entropy} & = 1.175 \\
    \text{accuracy}       & = 0.72
\end{align}


โดยผลการทำนายของ model  จาก Blind Test ได้เป็น Confusion matrix ดังรูป \ref{fig:Confusion matrix}
\begin{figure}[h]
    \begin{center}

        \includegraphics[scale=0.1]{pic/model/blind_pic_4_ccm.png}
    \end{center}

    \caption[Confusion matrix]{Confusion matrix}
    \label{fig:Confusion matrix}
\end{figure}


\newpage

\section{ผลลัพธ์จากการฝึกสอน}
การทดสอบ แบบชุดข้อมูล (Batch Detection) สําหรับการทดสอบ Model โดยได้ทำการทดลอง ฝึกสอน Model 
\par โดยเป็นผลลัพธ์จาก Model ที่ดีที่สุดจากทั้ง 4 Fold
ที่ฝึกสอนด้วย จำนวนรอบ(epoch) 100 และ 200  , batch size เป็น 32 และ 64 , และมีการใช้ Dropout Layer ที่ dropout rate เป็น 0 และ 0.5
ได้ผลลัพธ์ ดังตาราง \ref{tab:model_train_performance} และผลลัพธ์จากการทดสอบด้วย Blind Dataset ดังตาราง \ref{tab:model_test_performance}

\begin{table}[ht]
    \centering
    \caption{ผลลัพธ์จากการทดสอบ Model  โดยใช้ Train Dataset}
    \label{tab:model_train_performance}

    \begin{tabular}{|l|l|l|l|l|l|l|l|}
        \hline
        \multirow{2}{*}{Model name} & \multirow{2}{*}{Batch Size} & \multirow{2}{*}{Epoch} & \multirow{2}{*}{Dropout Rate} & \multicolumn{4}{c|}{Metrics}                                                    \\ \cline{5-8}
                                    &                             &                        &                               & Accuracy                     & Precision      & Recall         & F1-Score       \\ \hline
        64-100                      & 64                          & 100                    & -                             & 0.605                        & 0.483          & 0.651          & 0.555          \\ \hline
        64-100-5                    & 64                          & 100                    & 0.5                           & 0.673                        & 0.391          & 0.823          & 0.530          \\ \hline
        64-200                      & 64                          & 200                    & -                             & 0.674                        & 0.546          & 0.767          & 0.638          \\ \hline
        64-200-5                    & 64                          & 200                    & 0.5                           & 0.698                        & 0.481          & 0.837          & 0.611          \\ \hline
        32-100                      & 32                          & 100                    & -                             & 0.734                        & 0.572          & 0.797          & 0.666          \\ \hline
        32-100-5                    & 32                          & 100                    & 0.5                           & 0.641                        & 0.469          & 0.766          & 0.582          \\ \hline
        32-200                      & 32                          & 200                    & -                             & \textbf{0.859}               & \textbf{0.625} & 0.806          & \textbf{0.704} \\ \hline
        32-200-5                    & 32                          & 200                    & 0.5                           & 0.766                        & 0.559          & \textbf{0.875} & 0.682          \\ \hline
    \end{tabular}
\end{table}


\begin{table}[ht]
    \centering
    \caption{ผลลัพธ์จากการทดสอบ Model  โดยใช้ Blind Dataset}
    \label{tab:model_test_performance}
    \begin{tabular}{|l|l|l|l|l|l|l|l|}

        \hline
        \multirow{2}{*}{Model name} & \multirow{2}{*}{Batch Size} & \multirow{2}{*}{Epoch} & \multirow{2}{*}{Dropout Rate} & \multicolumn{4}{c|}{Metrics}                                                    \\ \cline{5-8}
                                    &                             &                        &                               & Accuracy                     & Precision      & Recall         & F1-Score       \\ \hline
        64-100                      & 64                          & 100                    & -                             & 0.551                        & 0.318          & 0.412          & 0.359          \\ \hline
        64-100-5                    & 64                          & 100                    & 0.5                           & 0.492                        & 0.267          & 0.254          & 0.260          \\ \hline
        64-200                      & 64                          & 200                    & -                             & 0.521                        & 0.365          & \textbf{0.591} & 0.451          \\ \hline
        64-200-5                    & 64                          & 200                    & 0.5                           & 0.621                        & 0.402          & 0.342          & 0.370          \\ \hline
        32-100                      & 32                          & 100                    & -                             & \textbf{0.647}               & 0.423          & 0.293          & 0.346          \\ \hline
        32-100-5                    & 32                          & 100                    & 0.5                           & 0.572                        & 0.277          & 0.369          & 0.316          \\ \hline
        32-200                      & 32                          & 200                    & -                             & 0.628                        & \textbf{0.459} & 0.343          & \textbf{0.393} \\ \hline
        32-200-5                    & 32                          & 200                    & 0.5                           & 0.638                        & 0.382          & 0.381          & 0.381          \\ \hline
    \end{tabular}
\end{table}

สําหรับการวัดผลได้แก่ Accuracy, Precision, Recall, และ F1-Score
โดยโมเดลส่วนใหญ่ในการ Validation model ก็ให้ผลลัพธ์ Accuracy ที่ประมาน 70-80\%  ดังตาราง \ref{tab:model_train_performance}
แต่ในส่วนของการ Blind Test จะได้ Accuracy อยู่ที่ประมาณ 50-60\% เท่านั้น ดังตาราง \ref{tab:model_test_performance}







\newpage
\section{ผลการทำงานของระบบแอปพลิเคชันบนโทรศัพท์มือถือ}



จากการปฏิบัติงาน สามารถพัฒนา Application สำหรับ ลูกค้า ได้เสร็จสมบูรณ์ โดยสามารถแสดงผลการทำงานของหน้า Application แบ่งตามแต่ละหน้า ได้ดังนี้
\begin{enumerate}
    \item หน้าแสดงสินค้า และ รายละเอียดของสินค้า (Stock page)
    \item หน้าแสกนสินค้า หรือ อ่าน barcode โดยใช้กล้อง (Streaming page)
    \item หน้าลงชื่อเข้าใช้งาน และ ลงทะเบียนเข้าใช้งาน (Login / Register page)
    \item หน้าการตั้งค่าผู้ใช้  (Profile / Edit profile page )
    \item หน้าแสดงตะกร้าสินค้าที่ผู้ใช้เลือก (cart page)
    \item หน้าแสดงประวัติการซื้อสินค้าของลูกค้า (History page)
\end{enumerate}

\newpage


\subsection{หน้าแสดงสินค้า และ รายละเอียดของสินค้า (Stock page)}
สามารถกดดูข้อมูลสินค้าที่เลือก ดูจำนวนคงเหลือ ราคา คำอธิบายสินค้า และ สามารถค้นหาสินค้าได้\\
และสามารถเพิ่มสินค้าเข้าไปยัง ตระกร้าสินค้า (cart) ได้\\
\begin{center}
    \begin{tabular}{c@{\hspace{3cm}}c}

        \includegraphics[scale=0.3]{pic/moblie/stock.jpg} & \includegraphics[scale=0.3]{pic/moblie/search_products.jpg} \\
        Stock                                             & Search products
    \end{tabular}
\end{center}

\begin{center}
    \begin{tabular}{c@{\hspace{3cm}}c}
        \includegraphics[scale=0.3]{pic/moblie/stock_detail.jpg} & \includegraphics[scale=0.3]{pic/moblie/cart.png} \\
        Product detail                                           & Add to cart
    \end{tabular}
\end{center}



\newpage
\subsection{หน้าแสกนสินค้า หรือ อ่าน barcode โดยใช้กล้อง (Streaming page)}
ผู้ใช้สามารถใช้ กล้องของ smart phone ส่องไปที่สินค้าเพื่อทำการ Streaming รูปของสินค้า ไปยัง Classification Server
หรือ สามารถ เลือกที่จะแสกน barcode ของสินค้า เพื่อทำการเพิ่มสินค้าเข้าตะกร้า (cart)ได้
\begin{center}
    \begin{tabular}{c@{\hspace{3cm}}c}
        \includegraphics[scale=0.3]{pic/moblie/camera_ai.png} & \includegraphics[scale=0.3]{pic/moblie/camera_barcode.png} \\
        Camera streaming                                      & Camera barcode
    \end{tabular}
\end{center}


\subsection{หน้าลงชื่อเข้าใช้งาน และ ลงทะเบียนเข้าใช้งาน (Login / Register  page)}
สำหรับหน้านี้  โดยสามารถลงชื่อเข้าใช้งาน หรือ สามารถสมัครสมาชิกได้ โดยการใช้ Email และ Password\\

\begin{center}
    \begin{tabular}{c@{\hspace{3cm}}c}

        \includegraphics[scale=0.3]{pic/moblie/login.jpg} & \includegraphics[scale=0.3]{pic/moblie/signup.jpg} \\
        Login                                             & Signup
    \end{tabular}
\end{center}




\newpage
\subsection{หน้าการตั้งค่าผู้ใช้   (Profile page / Edit profile page)}
สำหรับหน้านี้  เมื่อผู้ใช้ทำการเข้าสู้ระบบ สามารถทำการก้ไขข้อมูลส่วนตัวของผู้ใช้ได้  และ สามารถเติมเงินเสมือน (Mock Credit) เพื่อใช้ในการซื้อสินค้าได้\\
\begin{center}
    \begin{tabular}{c@{\hspace{3cm}}c}

        \includegraphics[scale=0.3]{pic/moblie/profile.jpg} & \includegraphics[scale=0.3]{pic/moblie/edit.jpg} \\
        Profile                                             & Edit profile
    \end{tabular}
\end{center}



\subsection{หน้าแสดงตะกร้าสินค้าที่ผู้ใช้เลือก (cart page)  }
สินค้าที่ผู้ใช้ได้ทำการเลือกเพิ่มเข้ามาในตะกร้า ไม่ว่าจะมาจาก Stock page, Streaming page จะถูกสรุปเป็นรายการและจำนวนเงินที่ผู้ใช้ต้องทำการจ่าย
โดยผู้ใช้สามารถกด checkout และตัดเงินจาก Credit\\
\begin{center}
    \begin{tabular}{c@{\hspace{3cm}}c}
        \includegraphics[scale=0.3]{pic/moblie/cart.png} & \includegraphics[scale=0.3]{pic/moblie/checkout.png} \\
        Cart                                             & Checkout
    \end{tabular}
\end{center}


\subsection{หน้าแสดงประวัติการซื้อสินค้าของลูกค้า (History page) }
โดยเมื่อทำการ checkout เสร็จแล้ว  Credit ก็จะถูฏตัด รวมถึง Stock ของสิ้นค้าภายในร้าน โดยประวัติการซื้อของผู้ใช้ก็จะสามารถดูได้ภายใน Application
\begin{center}
    \begin{tabular}{c@{\hspace{3cm}}c}
        \includegraphics[scale=0.3]{pic/moblie/profile.jpg} & \includegraphics[scale=0.3]{pic/moblie/history.jpg} \\
        Purchase history                                    & Purchase history
    \end{tabular}
\end{center}






\newpage
\section{ผลการ checkout}
จากการทดลองและบันทึกผลของเรา ทำการบันทึกข้อมูลและค่าเฉลี่่ยได้ดังนี้
(ขึ้นอยู่กับผู้ใช้งานแต่ละคน)

\begin{table}[h]
    \centering
    \caption{Camera classification process}
    \begin{tabular}{|p{5cm}|p{4cm}|p{4cm}|}
        \hline
        \textbf{Camera Classification} & \textbf{ซื้อของ 1 ชิ้น (วินาที)} & \textbf{ซื้อของ 5 ชิ้น (วินาที)} \\
        \hline
        Camera streaming               & 2-5 s                      & 10-25 s                    \\
        Image processing               & 1-2 s                      & 5-10 s                     \\
        Payment transaction            & 2 s                        & 2 s                        \\
        \hline
        Total Time Usage               & 5-9 s                      & 17-37 s                    \\
        \hline
    \end{tabular}
\end{table}

\begin{table}[h]
    \centering
    \caption{Camera barcode process}
    \begin{tabular}{|p{5cm}|p{4cm}|p{4cm}|}
        \hline
        \textbf{Barcode payment} & \textbf{ซื้อของ 1 ชิ้น (วินาที)} & \textbf{ซื้อของ 5 ชิ้น (วินาที)} \\
        \hline
        Camera capture           & 2-5 s                      & 10-25 s                    \\
        Barcode processing       & 0.5 s                      & 2.5 s                      \\
        Payment transaction      & 2 s                        & 2 s                        \\
        \hline
        Total Time Usage         & 2.5-7 s                    & 14.5-29.5 s                \\
        \hline
    \end{tabular}
\end{table}

\begin{table}[h]
    \centering
    \caption{Line Checkout process}
    \begin{tabular}{|p{5cm}|p{4cm}|p{4cm}|}
        \hline
        \textbf{Line Checkout Cashier} & \textbf{ซื้อของ 1 ชิ้น(seconds)} & \textbf{ซื้อของ 5 ชิ้น (seconds)} \\
        \hline
        Waiting in line                & 30-120 s                     & 30-120 s                      \\
        Cashier processing             & 10-20 s                      & 20-40 s                       \\
        Payment transaction            & 5-10 s                       & 5-10 s                        \\
        \hline
        Total Time Usage               & 45-150 s                     & 55-170 s                      \\
        \hline
    \end{tabular}
\end{table}

\newpage
\section{ผลการทำงานของระบบเว็บไซต์สำหรับร้านค้า}
จากการปฏิบัติงาน สามารถพัฒนาระบบระบบเว็บไซต์สำหรับร้านค้าได้เสร็จสมบูรณ์ โดยสามารถแสดงผลการทำงานของหน้าเว็บไซต์แบ่งตามแต่ละหน้าของเว็บไซต์ได้ดังนี้
\begin{enumerate}
    \item หน้าลงชื่อเข้าใช้งาน (Login)
    \item หน้าการตั้งค่าร้านค้า และผู้ดูแลร้านค้า (Store Settings)
    \item หน้าการจัดการสินค้า (Stock Manager)
    \item หน้าแสดงรายงานยอดขายสินค้า (Selling Report)
    \item หน้าแสดงข้อมูลลูกค้า (Customers)
    \item หน้าแสดงประวัติการซื้อสินค้าของลูกค้า (Purchase History)
\end{enumerate}

\subsection{หน้าลงชื่อเข้าใช้งาน (Login)}
สำหรับหน้านี้ จะเป็นหน้าแรกของการเข้าใช้งานระบบ โดยสามารถลงชื่อเข้าใช้งานได้เท่านั้น แต่จะไม่สามารถลงทะเบียนได้ เนื่องจากผู้ใช้งานระบบ ต้องเป็นผู้ดูแลระบบร้านค้า หรือพนักงานของร้าน ซึ่งจะถูกเพิ่มโดย Super Administer ซึ่งอาจเป็นเจ้าของร้าน หรือผู้จัดการร้าน โดยผู้ใช้งานสามารถขอข้อมูลรหัสผ่านจาก Super Administer โดยตรงเพื่อนำมาใช้เข้าสู่ระบบ\\
{
\includegraphics[scale=0.35]{pic/ui/w26.png}
}\\
{
\includegraphics[scale=0.35]{pic/ui/w27.png}
}\\
{
\includegraphics[scale=0.35]{pic/ui/w1.png}
}\\
{
\includegraphics[scale=0.35]{pic/ui/w2.png}
}\\

\subsection{หน้าการตั้งค่าร้านค้า และผู้ดูแลร้านค้า (Store Settings)}
สำหรับหน้านี้ผู้ใช้งานสามารถตั้งค่าชื่อร้าน สถานที่ตั้ง และประเภทร้านค้าได้ รวมถึงเป็นหน้าการจัดการผู้ดูแลระบบ โดยการแสดงผลสามารถแบ่งได้เป็น 2 กรณีตามสถานะของผู้ใช้งาน ได้แก่
\begin{enumerate}
    \item Super Administer: สามารถดูรายชื่อ เพิ่ม หรือลบผู้ดูแลระบบร้านค้า (Administer) ได้ และสามารถแก้ไขมูลสถานะ และบทบาทของผู้ดูแลระบบร้านค้า และแก้ไขข้อมูลของตนเองได้
    \item Administer: สามารถดู และแก้ไขข้อมูลของตนเองได้เท่านั้น
\end{enumerate}
{
\includegraphics[scale=0.35]{pic/ui/w3.png}
}\\
\subsection{หน้าการจัดการสินค้า (Stock Manager)}
สำหรับหน้านี้จะแสดงรายการสินค้าทั้งหมดแยกตามหมวดหมู่ พร้อมข้อมูลทั้งหมด ได้แก่ ชื่อสินค้า รูปสินค้า ข้อมูลสินค้า ราคา จำนวนในคลัง และประวัติคลังสินค้า โดยสามารถทำการเพิ่ม ลบ และแก้ไขข้อมูลทั้งหมดได้ ดังนี้\\
{
\includegraphics[scale=0.35]{pic/ui/w4.png}
}\\
{
\includegraphics[scale=0.35]{pic/ui/w5.png}
% \includegraphics[scale=0.35]{pic/ui/w6.png}
}\\
{
\includegraphics[scale=0.35]{pic/ui/w7.png}
}\\
{
\includegraphics[scale=0.35]{pic/ui/w8.png}
}\\
{
\includegraphics[scale=0.35]{pic/ui/w9.png}
}\\
{
\includegraphics[scale=0.35]{pic/ui/w10.png}
}\\
{
\includegraphics[scale=0.35]{pic/ui/w11.png}
}\\
{
\includegraphics[scale=0.35]{pic/ui/w12.png}
}\\
{
\includegraphics[scale=0.35]{pic/ui/w13.png}
}\\
{
\includegraphics[scale=0.35]{pic/ui/w14.png}
}\\
\subsection{หน้าแสดงรายงานยอดขายสินค้า (Selling Report)}
เมื่อโหลดหน้าใหม่จะแสดงข้อมูลยอดขายของวันที่ปัจจุบันผ่านการ์ดข้อมูล แผนภูมิ และตาราง ในโหมดรายวัน ตามสินค้าทั้งหมด สามารถเลือกวันที่ที่ต้องการ และเปลี่ยนการแสดงผลตามรายวัน รายสัปดาห์ รายเดือน และรายปี ตามหมวดหมู่ของสินค้า หรือตามสินค้าทั้งหมดได้\\
{
\includegraphics[scale=0.35]{pic/ui/w15.png}
}\\
{
\includegraphics[scale=0.35]{pic/ui/w16.png}
}\\
{
\includegraphics[scale=0.35]{pic/ui/w17.png}
}\\
{
\includegraphics[scale=0.35]{pic/ui/w18.png}
}\\
{
\includegraphics[scale=0.35]{pic/ui/w19.png}
}\\
{
\includegraphics[scale=0.35]{pic/ui/w20.png}
}\\
{
\includegraphics[scale=0.35]{pic/ui/w21.png}
}\\
{
\includegraphics[scale=0.35]{pic/ui/w22.png}
}\\
\subsection{หน้าแสดงประวัติการซื้อสินค้าของลูกค้า (Purchase History)}
สำหรับหน้านี้จะแสดงรายการการซื้อของลูกค้าแต่ละครั้งในรูปแบบการ์ด ประกอบไปด้วยชื่อ วัน และเวลาที่ซื้อ และจำนวนเงินที่จ่าย เมื่อกดที่การ์ดจะแสดงรายการสินค้าที่ซื้อ จำนวนที่ซื้อ และราคาของสินค้าแต่ละชิ้น\\
{
\includegraphics[scale=0.35]{pic/ui/w23.png}
}\\
\subsection{หน้าแสดงข้อมูลลูกค้า (Customers)}
สำหรับหน้านี้จะแสดงข้อมูลของลูกค้าแต่ละคนผ่านตาราง ได้แก่ ชื่อลูกค้า วันเกิด วันที่ลงทะเบียน และอีเมล์ และผ่านการ์ดข้อมูล ได้แก่จำนวนลูกค้าทั้งหมด จำนวนลูกค้าที่ลงทะเบียนใหม่ใน 7 วันที่ผ่านมา\\
{
\includegraphics[scale=0.35]{pic/ui/w24.png}
}\\


% ในส่วนของการออกแบบส่วนสื่อประสานกับผู้ใช้ (GUI) ของ Website Dashboard นั้น ได้มีการพัฒนาตามที่ออกแบบไว้ แต่มีการปรับเปลี่ยนดีไซน์ และสีของหน้าเล็กน้อย แสดงผลลัพธ์การใช้งานได้ดังนี้
% \begin{enumerate}
%     \item Get started page: หน้าการเข้าสู่ระบบ หรือลงทะเบียนเข้าใช้งาน
%     \item Login page: หน้าการเข้าสู่ระบบ
%     \item Home page:  หน้าแสดงเมนูของฟังก์ชันต่าง ๆ
%     \item Menu bar:  แสดงเมนูของฟังก์ชันต่าง ๆ
%     \item Stock manager page: หน้าการจัดการคลังสินค้า
%     \item Stock manager page > Add new category: หน้าต่างการเพิ่มประเภทสินค้าใหม่
%     \item Stock manager page > Delete category: หน้าต่างการลบประเภทสินค้า
%     \item Stock manager page > Add new product: หน้าต่างการเพิ่มสินค้าใหม่
%     \item Stock manager page > Get product stock history: หน้าต่างการเพิ่มติดตามประวัติคลังสินค้า
%     \item Stock manager page > Delete product: หน้าต่างการลบสินค้า
%     \item Stock manager page > Edit product: หน้าต่างการแก้ไขสินค้า
%     \item Selling report page: หน้าการแสดงผลข้อมูลยอดขายตามรายวัน รายเ��ือน และรายปี โดยสามารถดูตามหมวดหมู่ของสินค้า หร���อแยกตามสินค้าหนึ่ง ๆได้
%     \item Selling report page > Date Time Picker: เลือกวันที่ต้องการติดตามรายงานยอดขาย
%     \item Selling report page > Graph: แสดงรายงานยอดขายผ่านแผนภาพ
%     \item Selling report page > Table: แสดงรายงานยอดขายผ่านตารางจากมากไปน้อย
%     \item Purchase History page: หน้าการติดตามประวัติการซื้อของลูกค้า
%     \item Customer page: หน้าแสดงข้อมูลของลูกค้าที่ลงทะเบียนผ่านโมบายแอปพลิเคชัน
% \end{enumerate}


% \end{center}

