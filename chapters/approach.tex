\chapter{\ifproject%
\ifenglish Project Structure and Methodology\else โครงสร้างและขั้นตอนการทำงาน\fi
\else%
\ifenglish Project Structure\else โครงสร้างของโครงงาน\fi
\fi
}

ในบทนี ้ จะกล่าวถึงหลักการ,  การนําทฤษฎีที่เกี่ยวข้องมาประยุกต์ใช้  และการออกแบบของระบบ


\makeatletter

% \renewcommand\section{\@startsection {section}{1}{\z@}%
%                                    {13.5ex \@plus -1ex \@minus -.2ex}%
%                                    {2.3ex \@plus.2ex}%
%                                    {\normalfont\large\bfseries}}

\makeatother
%\vspace{2ex}
% \titleformat{\section}{\normalfont\bfseries}{\thesection}{1em}{}
% \titlespacing*{\section}{0pt}{10ex}{0pt}
\section{โครงสร้างของระบบ}
% \ref{fig:Overall project structure}
\begin{figure}[h]
  \begin{center}
  % \includegraphics{pic/webrtc.png}
  \vspace{0.5cm}\includegraphics[scale=0.5]{pic/overall_2.png}
  \end{center}
  
  \caption[Overall project structure]{Overall project structure}
  \label{fig:Overall project structure}
  \end{figure}
\section{เตรียมชุดข้อมูลฝึกสอน}
ข้อมูลที่ใช้ในการ train mode โดยมีสินค้าประมาณ 100 ชนิด โดยจัดเก็บข้อมูลใช้กล้องมือถือ ในการถ่ายภาพในมุมต่างๆ
ของสินค้าชนิดนั้นๆ ตามมุมต่างๆ จำนวนชนิดละ N รูป โดยจัดเก็บใน  และทำการดึงข้อมูลมา train ผ่าน Google Colab
 โดยโครงสร้างการเก็บข้อมูลจะเป็นดังรูป

\begin{center}
  \includegraphics[scale=0.35]{pic/st.png}
\end{center}

\section{การเพิ่ม traing data}
โดยสินค้า 1 ชนิด ทำการถ่ายภาพ 6 รูป ในมุมที่แตกต่างกัน 
โดยในแต่ละ 1 รูปภาพที่ถ่าย จะแปลงเป็นรูปภาพ RGB ขนาด 224x224 pixel
และในแต่ละรูปเพื่อให้มี train dataset จำนวนมาก นำไปหมุนและกลับด้าน ด้วยมุม -20,-15,-10,-5,0,5,10,15,20 องศา
โดย 1 รูปภาพผ่านการ generate datasets จะกลายเป็น 18 รูปภาพซึ่งมีความแตกต่างกันเล็กน้อย
\begin{figure}[h]
  \begin{center}
  % \includegraphics{pic/webrtc.png}
  \includegraphics[scale=0.4]{pic/genmore.png}
  \end{center}
  
  \caption[Dataset generator]{Dataset generator}
  \label{fig:Dataset generator}
  \end{figure}

  \newpage
\subsection{Model Architecture}
โดยโมเดลในโครงงานนี้จะใช้ Xception pre-trained มาใช้ในการแยกคุณลักษณะเด่นของรูปภาพ  และสร้างโมเดลมาต่อท้ายเพื่อ 
เรียนรู้ลักษณะเด่นจากที่ Xception ทำการแยกออกมาได้
\begin{center}
  \includegraphics[scale=0.45]{pic/model.png}
\end{center}
  
โดยต่อท้ายด้วย fully connected node ที่เรียกว่า Dense layer ซึ่งทำหน้าที่เป็น classifier 
 โดยมี output layer ที่มีจำนวน node เท่ากับจำนวนสินค้า สำหรับการ classify ชนิดของ
  products จาก รูปภาพ
 
% \section{Product database}
% Section 2 text.

 

 
% \subsection{Transfer Learning}

Subsection 1 text

\section{classification products}
จากรูปภาพใน 48 class ผ่านการ generate datasets จะมี dataset ทั้งหมด 4432 sample
 ทำการแบ่งเป็น train 3546 sample และ  886 sample สำหรับการ evaluate 
โดยจาก train 3546 แบ่ง 50\% สำหรับการ validation ในระหว่างการ train model

\par ผลลัพธ์ จากการ train \& validation ด้วย 3546 sample เป็นจำนวน 40 Epoch
\begin{figure}[h]
  \begin{center}
  % \includegraphics{pic/webrtc.png}
  \includegraphics[scale=0.45]{pic/train.png}
  \end{center}
  
  \caption[Train results]{Train results}
  \label{fig:Train results}
  \end{figure}

  % โดยเมื่อนำ evaluate มาหา cross_entropy 0.23343226313591003, 0.944695234298706 accuracy

  และทำการ save model ที่มีความแม่นยำระดับนึง
   สำหรับเป็น service ในการ classify products ของ application ผ่าน aiortc  และเว็บ WebRTC

  % score (cross_entropy, accuracy):
%  [0.23343226313591003, 0.944695234298706]

% \begin{figure}[h]
% \begin{center}
% % \includegraphics{pic/webrtc.png}
% \includegraphics[scale=0.5]{pic/webrtc.png}
% \end{center}

% \caption[webrtc structure]{webrtc structure}
% \label{fig:webrtc structure}
% \end{figure}
\newpage

\section{การพัฒนา application }
ใช้ Flutter ในการสร้าง application ในส่วนของผู้ใช้โดยใช้ service ของ WebRTC ติดต่อกับ classification server ที่ได้ทำการ train ไว้แล้ว
โดยจะ application จะทำการ streaming ภาพสินค้าที่ลูกคต้าถ่าย ผ่าน WebRTC ไปยัง server ของทางร้าน ที่มี model classifier อยู่
และ server ของทางร้านก็ classification ว่าเป็นสินค้าชนิดใด และตอบกับมายัง application
\subsection{Requirement Specification}  

\begin{enumerate}
  \item ผู้คนทั่วไปสามารถลงทะเบียนเข้าใช้งานได้ผ่าน Email , SSO , หมายเลขโทรศัพท์มือถือ
  \item สามารถเชื่อมต่อกับข้อมูลร้านค้าได้ผ่านการแสกนคิวอาร์โค้ดเข้าใช้งานร้านค้า
  \item สามารถใช้กล้องโทรศัพท์มือถือแสกนสินค้าเพื่อสตรีมภาพ Classification server เพื่อทำการดึงข้อมูลสินค้าจาก Server ของร้านค้ามาแสดงผลบนแอพลิเคชัน
  \item สามารถเพิ่มหรือลดสินค้าในตะกร้าได้
  \item สามารถทำการชำระเงินได้ผ่าน Payment gateway ในตัวแอพลิเคชัน
  \item สามารถ Checkout จากร้านค้าผ่านการแสกนคิวอาร์โค้ดออกจากร้านค้า
 
\end{enumerate}


\section{การเตรียมฐานข้อมูล}
ในการเก็บฐานข้อมูลจะแบ่งเป็นฐานข้อมูลสองตัว โดยใช้ Supabase 
ในการสร้างโครงการ และจัดฐานข้อมูลทั้งหมดแบบ SQL ได้แก่ 
\subsection{ฐานข้อมูลของระบบ}
สำหรับเก็บข้อมูลร้านค้าที่เข้าร่วม และประวัติการซื้อสินค้าของผู้ใช้งานแอพลิเคชันมือถือ ซึ่งมี Database schema ดังนี้ 
\begin{figure}[h]
\begin{center}
  % \includegraphics{pic/webrtc.png}
  \includegraphics[scale=0.3]{pic/datamobile.png}
  \end{center}
  
  \caption[Mobile Application Database Schema]{Mobile Application Database Schema}
  \label{fig:Mobile Application Database Schema}
  \end{figure}


  
\subsection{ฐานข้อมูลในแต่ละร้านค้า}
สำหรับเก็บข้อมูลสินค้า และประวัติยอดขายของร้านค้า โดยร้านค้าแต่ละร้านจะมีฐานข้อมูลเป็นของตนเองเพื่อใช้งานกับ Server ของร้านนั้น ๆ
 โดยตรง ซึ่งจะมี Database schema ดังนี้ 

 \begin{figure}[h]
  \begin{center}
    % \includegraphics{pic/webrtc.png}
    \includegraphics[scale=0.3]{pic/dataweb.png}
    \end{center}
    
    \caption[Store Website Dashboard Database Schema]{Store Website Dashboard Database Schema}
    \label{fig:Store Website Dashboard Database Schema}
    \end{figure}
  
\section{การพัฒนาweb dashboard }


ออกแบบ UI/UX ของเว็บไซต์และ application ด้วย Figma
หลังจากการออกแบบ และจัดการตั้งค่าฐานข้อมูลเสร็จแล้ว
 ก็จะพัฒนาในส่วนของเว็บไซต์ทางฝั่งร้านค้าโดยใช้ Supabase ในการโฮสติ้งเว็บไซต์ 
 และเชื่อมต่อกับฐานข้อมูลที่ได้ตั้งค่าไว้ โดยใช้ Flask framework 
 ในการจัดการ API และ Python ในการจัดการระบบ Backend ของเว็บไซต์โดยใช้ SQLAlchemy
  ในการดึงข้อมูลจากฐานข้อมูล Requirement Specification ดังนี้

  \begin{enumerate}
    \item สามารถเข้าใช้งานได้ผ่านการยืนยันตัวตนเป็น Administers เท่านั้น โดยสามารถมีได้ 1-5 คน
    \item สามารถดูคลังสินค้า และแก้ไขข้อมูลสินค้าในแต่ละชนิดได้แบบเรียลไทม์
    \item สามารถดูสถิติยอดขายสินค้าได้ทั้งแบบรายวัน รายเดือน และรายปี โดยแบ่งได้ 2 แบบ คือตามชนิดสินค้า และประเภทสินค้า
    \item สามารถกดูประวัติการขายตามออเดอร์ของลูกค้าแต่ละคนได้ และดูข้อมูลของแต่ละออร์เดอร์ได้
\end{enumerate}



 

\section{การทดสอบการทำงานของซอร์ฟแวร์}
การทดสอบการทำงานของระบบ สามารถแบ่งเป็นขั้นตอนดังนี้
\subsection{Unit testing}
การทดสอบความถูกต้องของการทำงานในแต่ละฟังก์ชันหลักของระบบแยกกัน โดยยังไม่รวมแต่ละ Component เข้าด้วยกัน ซึ่งได้แก่
\begin{enumerate}
  \item Classification system
  \item Website dashboard
  \item Mobile Application
\end{enumerate}

\subsection{Integration testing}
การทดสอบการทำงานเมื่อรวมระบบย่อยทั้งหมดเข้าด้วยกัน โดยหลัก ๆ
 จะทดสอบในเรื่อง API ว่ามีการรับส่งข้อมูลดุถูกต้องหรือไม่ และทำงานโดยรวมได้ถูกต้องทั้งหมดหรือไม่
\subsection{System testing}
การทดสอบระบบซึ่งแต่ละโมดูลข้างต้นจะถูกรวม และทดสอบเป็นกลุ่ม 
เพื่อประเมินความสอดคล้องของระบบว่าทำงานได้ตามที่กำหนดไว้หรือไม่
\subsection{Acceptance testing}
การทดสอบระบบโดยดูภาพรวมของการทำงาน ว่ามีการตอบสนองความต้องการของผู้ใช้ทั้งในส่วนของฟังก์ชันการทำงาน 
และประสิทธิภาพการทำงาน ว่าสอดคล้องกับลักษณะของความต้องการของซอฟต์แวร์หรือไม่ 
โดยใช้การทดสอบแบบ Functional testing (Black box testing)
\section{แผนภาพกระแสข้อมูล (Data Flow Diagram)}
\begin{figure}[h]
  \begin{center}
  % \includegraphics{pic/webrtc.png}
  \includegraphics[scale=0.25]{pic/dataflow-lv0.png}
  \end{center}
  
  \caption[Data Flow Diagram]{Data Flow Diagram}
  \label{fig:Data Flow Diagram}
  \end{figure}
  

% \subsection{The Black Kitten}
  % One thing was certain, that the WHITE kitten had had nothing to
% do with it:---it was the black kitten's fault entirely



% ~\cite{aiw}. 




%  For the
% white kitten had been having its face washed by the old cat for
% the last quarter of an hour (and bearing it pretty well,
% considering); so you see that it COULDN'T have had any hand in
% the mischief.

%   The way Dinah washed her children's faces was this:  first she
% held the poor thing down by its ear with one paw, and then with
% the other paw she rubbed its face all over, the wrong way,
% beginning at the nose:  and just now, as I said, she was hard at
% work on the white kitten, which was lying quite still and trying
% to purr---no doubt feeling that it was all meant for its good.

%   But the black kitten had been finished with earlier in the
% afternoon, and so, while Alice was sitting curled up in a corner
% of the great arm-chair, half talking to herself and half asleep,
% the kitten had been having a grand game of romps with the ball of
% worsted Alice had been trying to wind up, and had been rolling it
% up and down till it had all come undone again; and there it was,
% spread over the hearth-rug, all knots and tangles, with the
% kitten running after its own tail in the middle.

% \subsection{The Reproach}

  